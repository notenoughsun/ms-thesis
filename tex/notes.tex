\documentclass[conference]{IEEEtran}
%\IEEEoverridecommandlockouts
% The preceding line is only needed to identify funding in the first footnote. If that is unneeded, please comment it out.
\usepackage{cite}
\usepackage{amsmath,amssymb,amsfonts}
\usepackage{algorithmic}
\usepackage{graphicx}
\graphicspath{{images/}}
\DeclareGraphicsExtensions{.pdf,.png,.jpg}
\usepackage{textcomp}
\usepackage{xcolor}
\def\BibTeX{{\rm B\kern-.05em{\sc i\kern-.025em b}\kern-.08em
		T\kern-.1667em\lower.7ex\hbox{E}\kern-.125emX}}
\begin{document}

2) Many problems accept discretization. Not all of them.

\subsection{from l02 planning}

we apply the discretisation to our problem, and transform the global search task, to local graph search problem. The problem becomes much similar to HMC solutions and realisations.

The objective is to find a feasible solution to slam and planning problems. Why does the term feasible is applied here? We have to first define, what is infeasible solution, and which system is inappropriate for usage.

Considering the probabilistic approach, once we have some uncertainty in data, we need a way to reduce this uncertainty. So, the only requirement for the system, is that the covariance in localization problem can be decreased.

In our implementation, we perform integration of IMU data. The uncertainty increases quadratically.

After a series of integration steps, we have a new pose. The uncertainty can be decreased only after adding new information on human or location. The information we have are sensor measurements, building map, IMU data. For the extension of current aproach we may consider pose prediction methods, special approach of human step detection and different sesnor fusion models. Here we don’t consider this refinements and focus on localization and mapping method only for a given sensor model.

The dead-reckoning is the localization or pose update when the sensor data are not available. In common indoor localization sistem , as in [] traditional sensors are [] we model the localization with loosy signal connection of WiFi and BLE, we estimate how the spatial data of magnetic field and IMU PDR can increase the localization accuracy.

Given the proof that the spatial information of magnetic field has a significan effect for traditional PDR, we focus on this technology.

We have examples of products that are based mostly on magnetics field

There is no information of how the magnetic data can be collected in absence of precise location. We searched for a possible solutions to this task.

First, the task is to find a precise location which is usually called ground truth for training the system, creating the map of the location, for estimatimation of the algorithm performance.



\subsection{update 15.04}

MATHEMATICAL NOTATION OF AVAILABLE MEASUREMENTS IN WIRELESS COMMUNICATION SYSTEMS TOGETHER WITH APPROXIMATIVE NOISE STANDARD DEVIATIONS σ e . [HERE y IS THE ACTUAL NUMERICAL VALUE, h DENOTES A GENERAL NONLINEAR MODEL, p t IS THE SOUGHT POSITION, p DENOTES THE POSITION OF BASE STATION/ANTENNA NUMBER i AND e DENOTES MEASUREMENT NOISE WITH A PROBABILITY DENSITY FUNCTION p E (·).]

Fig. 2. Magnetic field as measured during the mapping along the lines parallel to . The y axis on the graphic corresponds to the number of the line and the x axis corresponds to the time of the measurement relative to the moments we started and finished following each line. Values are scaled and color- coded independently for each coordinate to better show spatial variations.

\subsection{from Unsupervised_Indoor_Localization}

The graph matching (or edge independent set) problem is the problem of finding a subset of graph edges, such that none of the edges share vertices. This problem is more interesting on bipartite graphs, where the matching is similar to assigning the vertices of one part of the graph to another, with no vertex appearing twice in the assignment. A specific case of this problem is the “maximum weighted bipartite matching”, which is defined as a matching on a bipartite graph with maximum sum of the weights of selected edges.

This problem is also known as the “assignment problem”. There are well-known algorithms for this problem such as the Hungarian algorithm (originally called by this name in [42]). Another method to solve this problem is to run the Bellman-Ford shortest path algorithm on an augmented graph.

problem: maximum weighted bipartite graph matching
approach: matching on a bipartite graph with maximum sum of the weights of selected edges

We now describe the problem formulation by modeling two graphs: the ground truth graph which is built offline, based on the building map, and the data graph, which is built in an unsupervised fashion, based on the readings obtained from users walking in the environment.

ground truth graph
building map

data graph
readings obtained from users walking in the environment

To find the loop closures and repetitive points, we use the information in the unlabeled RSS readings. The system is designed to recognize the overlapping parts of two walks in the building and merge them, based on the common RSS readings. This process has some similarity to the operation phase of localization systems with a data-set of labeled data, except that in this case, instead of finding the physical location of the user geologically, only matching unlabeled points (repetitive points) are found and merged. This process, can be considered as topological localization on the data graph, since no physical coordinate is obtained.

topological localization on the data graph - not exactly here, but we define a matching procedure for accurate matching


\subsection{Fast Iterative Alignment of Pose Graphs with Poor Initial Estimates}

Duckett et al. described an early nonlinear SLAM im- plementation [6] that uses Gauss-Seidel relaxation. However, they assumed absolute knowledge of the robot’s orientation, essentially making the problem linear. Frese, Larsson, and Duckett addressed that limitation and attempted to improve convergence speed in [7] with the Multi-Level Relaxation (MLR) algorithm. Multi-resolution methods are typically ap- plied to problems more spatially uniform than that of SLAM, but they report good results. MLR, given time, can generally find the exact minimum of the graph. Other nonlinear approaches include GraphSLAM [8], and Graphical SLAM [9]. Konolige proposes a method [10] for accelerating convergence by reducing the graph to poses that have a loop constraint attached, solving for the other nodes separately. This can save considerable CPU time, but requires the graph to have low connectivity.

Paskin’s Thin Junction Tree Filter [11] and Frese’s TreeMap [12] compute nonlinear map estimates, but their approaches require factorization of the joint probability density, which they achieve by ignoring small state correlations. These ap- proximations can result in noticeable map artifacts. A hybrid of linear and nonlinear solutions is Bosse’s Atlas [13], which uses linearized (EKF-based) submaps but stitches them together using nonlinear optimization. Nonlinear optimization algorithms have a rich history out- side the SLAM community. SLAM algorithms have typically limited themselves to Gauss-Seidel or Gradient Descent ap- proaches, but other approaches are commonly used in other fields. In particular, Stochastic Gradient D

Typically, different edges will lead to steps in different directions. SGD thus tends to hop around from one local minimum to another. SGD is also less likely to be caught in a long valley, since there is probably at least one edge which has a significant gradient. This edge will cause the state estimate to teleport to another part of the cost function, where the gradient may be more helpful. The distance that SGD travels for each edge is slowly decreased over time in order prevent oscillation. This makes it increasingly difficult for the state estimate to transition from one local minimum to another, with the result that it becomes increasing likely that SGD will get stuck in the most popular minimum (which is likely the global minimum). The degree to which SGD modulates its step size is known as the learning rate, and is analogous to the cooling rate in simulated annealing methods. A great number of learning rate schedules have been explored in the literature ([17] is an interesting example); the simplest are simple functions of iteration number, while others incorporate convergence rate and other information. The sim- plest strategy is to set the learning rate α ∝ 1/n, where n is the current iteration of the algorithm. The proportionality constant is determined, in our case, by examining the “stiffnesses” of the poses, as will be described in section III-E.


\end{document}